\chapter{Výsledky} \label{vysledky}

\section{Popis merania}

Keďže dynamické volanie funkcií cez introšpekciu je porovnateľné s~mechanizmom \texttt{dynamic} zo \Csharp{}, rozhodli sme sa porovnať ich rýchlosť. Oba spôsoby sme navyše porovnali s~virtuálnymi volaniami v~príslušnom jazyku.

Vytvorili sme teda štyri implementácie s~ekvivalentným chovaním. Ide o~visitor návrhový vzor. Existuje jedna trieda predka, päť jej potomkov, rozhranie visitoru a~jedna jeho implementácia. Na začiatku programu vytvoríme objekty potomkov. Pri každom sa náhodne vyberie, aký z~piatich typov objekt bude mať. Tieto objekty vložíme do zoznamu odkazov na predka. V~\Cpp{} použijeme \texttt{unique\_ptr}, v~\Csharp{} stačí obyčajná referencia. Potom prejdeme v~cykle všetky objekty, postupne ich navštevujeme virtuálnymi volaniami a~meriame čas. Potom prejdeme zoznam znovu, tentokrát s~dynamickým navštevovaním, a~meriame čas.

Implementácie v~\Cpp{} a~\Csharp{} sa nachádzajú v~prílohe pod názvami \texttt{Demo3} a~\texttt{Demo4}, respektíve.

Pri implementácii s~\PPreflection{} sme vytvorili dva varianty; jeden s~optimalizáciou pomocou dopredného vymenovania všetkých možných zoznamov typov argumentov, druhý bez. Pri neoptimalizovanej verzii mal zoznam objektov 100\,000 prvkov, pri všetkých zvyšných implementáciách sme prechádzali 10\,000\,000 prvkov.

Každú implementáciu sme spustili 15-krát, pričom sme zahodili prvé tri merania a~potom najrýchlejšie a~najpomalšie zo zvyšných 12. Ostalo teda desať meraní, z~ktorých sme vytvorili aritmetický priemer času na 10\,000\,000 volaní.

Programy boli spustené na počítači s~nasledovnými charakteristikami:
%$
\begin{table}[H]
\begin{tabular}{rl}
Operačný systém & Windows 11 Pro 64-bit, WSL2 Ubuntu 21.10 \\
Procesor & Intel Core i7 6700K @ 4\,GHz \\
RAM & 16\,GB @ 1333\,MHz \\
\end{tabular}
\end{table}

\section{Interpretácia výsledkov}

\begin{table}[H]
\centering
\begin{tabular}{l@{\hspace{1.2cm}}D{.}{,}{6.1}D{.}{,}{3.1}}
\toprule
\textbf{Metóda} & \mc{\textbf{Priemerný čas (ms)}} & \mc{\textbf{Nárast}} \\
\midrule
\Cpp{} \texttt{virtual}            &     82.6 & \mc{---} \\
\Csharp{} \texttt{virtual}         &    209.2 &     2.5 \\
\Csharp{} \texttt{dynamic}         &    813.3 &     3.9 \\
\PPreflection{} (s optimalizáciou) &   4691.5 &     5.8 \\
\PPreflection{}                    & 499400.0 &   106.4 \\
\bottomrule
\end{tabular}
\caption{Porovnanie rýchlosti \PPreflection{}}\label{tab2}
\end{table}

Ako prvé si môžeme všimnúť, že bez optimalizácie má \PPreflection{} takmer nepoužiteľne zlý výkon. To napovedá, že buď sme overload resolution implementovali veľmi slabo, alebo neexistuje implementácia, ktorá by spúšťala OR pri každom volaní a~stále mala rozumný výkon.

Po optimalizácii sú výsledky približne 106-krát lepšie, no keď sa pozrieme na ekvivalentnú implementáciu v~rovnakom jazyku, stále vidíme viac ako 50-násobné spomalenie. Nie je to fér porovnanie, pretože naša implementácia má v~skutočnosti viac schopností ako obyčajné virtuálne volanie, takže isté spomalenie sa očakáva.

Primerané by bolo porovnať ju s~\texttt{dynamic} v~jazyku \Csharp{}. Tam vidíme iba nárast o~faktor 5,8. Je teda možné, že ak by sme sa pokúsili opraviť ešte niekoľko nedostatkov v~implementácii knižnice, boli by sme schopní dosiahnuť podobné výsledky ako má \Csharp{} \texttt{dynamic}.

Druhé rozumné porovnanie je medzi nárastami rýchlosti vrámci jedného jazyka. V~\Csharp{} je \texttt{dynamic} 3,9-krát pomalší ako virtuálne volanie. \PPreflection{} je 56,8-krát pomalšie. Teda, ak by sa naša knižnica zrýchlila 14,5-krát, dostali by sme rovnakú cenu za dynamické volanie v~\Cpp{} ako v~\Csharp{}.
