\chapter{Použitie} \label{pouzitie}

Táto kapitola sa zaoberá vysvetlením použitia celého systému vytvoreného v~tejto práci. Vytvorený projekt a~príklady použitia sa nachádzajú v~prílohe, ktorej štruktúra je nasledovná:

\begin{figure}[H]
\dirtree{%
.1 /.
.2 Demo1/.
.2 Demo2/.
.2 Demo3/.
.2 Demo4/.
.2 PPreflection/.
.3 install/$^*$.
.4 PPreflection/.
.5 include/.
.5 generate\_metadata.sh.
.5 libPPreflection.a.
.5 PPreflector.so.
.3 configure.sh.
.3 build.sh.
.3 install.sh.
.3 clean.sh.
}
\caption{Štruktúra prílohy}
\end{figure}

\section{Preklad \PPreflection{} a~\PPreflector{}}

\subsection{Požiadavky}

Kvôli použitiu najnovšieho štandardu, \CppV{}, ktorý nie je v~prekladačoch naplno implementovaný, je zatiaľ možné preložiť projekt iba pomocou prekladača \GCCV{} s~argumentom \texttt{-std=c++20}. Keďže \PPreflector{} je plugin do \Clang{}, na jeho preklad musia byť nainštalované \textsf{LLVM 12} a~\ClangV{} vývojové balíčky. Na preklad používame \textsf{CMake} spolu s~\textsf{Ninja} build systémom. Skripty sú písane pre shell \textsf{Bash}.

\subsection{Postup}

V adresári projektu \path{/PPreflection/} sú pripravené skripty \path{configure.sh}, \path{build.sh} a~\path{install.sh} určené na automatický preklad. Najprv treba spustiť \path{configure.sh}. Ten pripraví potrebné súbory pre preklad. Potom spustenie \path{build.sh} preloží \PPreflection{} aj \PPreflector{}. Spustenie \path{install.sh} pripraví inštalačný adresár \path{install/}. Ten bude potrebný pri demonštračných projektoch.

\subsubsection{Pravdepodobná chyba v~\Clang{}}

\Clang{} pravdepodobne obsahuje chybu vo verzii 12, kde je v~jednom hlavičkovom súbore na dvoch miestach pri deklarácii konštruktoru šablónovej triedy použitý názov triedy vrátane šablónových argumentov. To \GCCV{} hlási pri preklade \PPreflector{} ako chybu. Riešenie je upraviť hlavičkový súbor a~odstrániť šablónové argumenty z~deklarácií.

\section{Preklad projektu s~introšpekciou}

\subsection{Požiadavky}

Pre použitie knižnice \PPreflection{} je nutný štandard \CppV{}. Kvôli použitiu najnovšieho štandardu je preklad projektov používajúcich introšpekciu zatiaľ možný iba s~\GCCV{}. Generácia metadát je vykonávaná spustením prekladača \ClangV{}.

\subsection{Príklad}

Predpokladajme projekt, kde máme deklarované namespace s~vnorenými triedami a~chceme vytvoriť funkciu, ktorá pre meno namespace vypíše jeho vnorené triedy.
%$
\begin{code}[fontsize=\footnotesize]
// print_namespace_types.hpp
namespace N { struct A {}; struct B {}; struct C {}; }
namespace M { struct D {}; struct E {}; }

#ifndef PPREFLECTOR_GUARD
void print_namespace_types(std::string_view name, std::ostream& out);
#endif

// print_namespace_types.cpp
#include "print_namespace_types.hpp"

#include "print_namespace_types.cpp.meta"

#ifndef PPREFLECTOR_GUARD
void print_namespace_types(std::string_view name, std::ostream& out)
{ /* introspection */ }
#endif

// main.cpp
#include "print_namespace_types.hpp"

int main()
{   print_namespace_types("N", std::cout);
    print_namespace_types("M", std::cout);
    return 0;
}
\end{code}
%$
Detaily implementácie funkcie \texttt{print\_namespace\_types} nie sú pre preklad dôležité.

Do súboru, ktorý používa introšpekciu, je potrebné vložiť pomocou \texttt{\#include} súbor s~metadátami, ktorého názov je celý názov súboru plus prípona \texttt{.meta}. \texttt{PPREFLECTOR\_GUARD} slúži na skrytie častí kódu, ktoré používajú namespace \texttt{std} alebo \PPreflection{}. Tieto dva namespace generátor metadát nerozpoznáva a~nemôže na vstupe dostať kód, ktorý ich používa.

Na preklad potrebujeme adresár \texttt{install/PPreflection/}, ktorý sa vytvára pri preklade knižnice. Obsahuje hlavičkové súbory, skript na generáciu metadát, statickú knižnicu \PPreflection{} a~plugin \PPreflector{}.  Potrebujeme ho skopírovať do adresára nášho projektu.

Ďalej potrebujeme vytvoriť súbory s~metadátami. Súbor \texttt{main.cpp} introšpekciu nepoužíva, takže stačí vytvoriť metadáta pre \path{print_namespace_types.cpp}. To urobíme pomocou skriptu \path{generate_metadata.sh}. Ten akceptuje ako prvý parameter cestu k~\PPreflector{} pluginu, v~tomto prípade \path{PPreflection/PPreflector.so}, a~ako druhý parameter cestu k~súboru, z~ktorého treba vytvoriť metadáta.

Keď máme vytvorené metadáta, môžeme začať preklad. Projekt potrebuje linkovať ako statickú knižnicu \PPreflection{}. Tiež potrebujeme nastaviť adresár pre objavenie hlavičkových súborov \path{include/}. Príklad v~\textsf{CMake}:
\begin{code}[fontsize=\footnotesize]
add_executable(ExecutableName "main.cpp" "print_namespace_types.cpp")
target_compile_features(ExecutableName PUBLIC cxx_std_20)
target_include_directories(ExecutableName PUBLIC
    "\${CMAKE_CURRENT_LIST_DIR}/PPreflection/include")
target_link_libraries(ExecutableName
    "\${CMAKE_CURRENT_LIST_DIR}/PPreflection/libPPreflection.a")
\end{code}

Tento príklad sa nachádza v~prílohe pod názvom \texttt{Demo1}.

\subsection{Demonštračné projekty}

V prílohe sú priložené tri demonštračné projekty využívajúce introšpekciu a~jeden projekt pre \textsf{.NET}.

\begin{itemize}[itemsep=0pt]
    \item \texttt{Demo1} ukazuje postup pri preklade s~introšpekciou.
    \item \texttt{Demo2} demonštruje schopnosti introšpekcie.
    \item \texttt{Demo3} meria výkon \Cpp{} a~\PPreflection{}.
    \item \texttt{Demo4} meria výkon \Csharp{}.
\end{itemize}

\texttt{Demo4} obsahuje skripty \path{build.sh} a~\path{clean.sh} určené na preklad a~vymazanie vytvorených súborov, respektíve. Spustiteľný súbor má meno \path{Demo4} a~vytvorí sa v~priečinku projektu spustením \path{build.sh}. Na preklad je potrebný \textsf{.NET 5.0}.

Pred prekladom prvých troch projektov je nutné skopírovať vygenerovaný adresár \path{/PPreflection/install/PPreflection/} do adresáru projektu.

Každý z~troch projektov používajúcich \PPreflection{} má vytvorené štyri skripty na ich automatický preklad. Súčasťou skriptu \path{build.sh} je vždy aj generácia metadát, takže na preklad stačí spustiť skripty v~tomto poradí: \path{configure.sh}, \path{build.sh}, \path{install.sh}. Potom sa v~priečinku projektu objaví spustiteľný súbor s~názvom zhodným s~názvom projektu. Na vymazanie súborov vytvorených pri preklade sa používa skript \path{clean.sh}.

\section{Dokumentácia}

V tejto časti sa nachádza podrobnejšia dokumentácia vybraných dôležitých tried z~\PPreflection{}.

Kompletnú dokumentáciu celého projektu je možné získať spustením príkazu \texttt{doxygen doxy\_config} v~priečinku \path{/PPreflection/}. Hlavná stránka \textsf{HTML} dokumentácie sa nachádza v~\path{/PPreflection/doc/html/index.html}.

\subsection{\texttt{dynamic\_reference}}

Dynamická referencia reprezentuje odkaz na ľubovoľný objekt alebo funkciu.

\subsubsection{Rozhranie}

\begin{code}[fontsize=\footnotesize]
dynamic_reference(const dynamic_reference&);
dynamic_reference(auto&&);
dynamic_reference& operator=(const dynamic_reference&);
auto get_type() const;
auto cast_unsafe(PP::concepts::type auto t) const -> PP_GET_TYPE(t)&&;
decltype(auto) visit(PP::concepts::type auto, auto&&) const;
decltype(auto) visit_ptr(PP::concepts::type auto, auto&&) const;
\end{code}

\subsubsection{\texttt{dynamic\_reference(const dynamic\_reference\& r)}}

Triviálny kopírovací konštruktor, vytvorí referenciu odkazujúcu tam ako \texttt{r}.

\subsubsection{\texttt{dynamic\_reference(auto\&\& r)}}

Vytvorí dynamickú referenciu na \texttt{r}. Je implicitný, takže nie je nutná explicitná konverzia napríklad pri volaní funkcie, ktorá akceptuje \texttt{dynamic\_reference}. To využívame pri všetkých funkciách \texttt{invoke}, ktoré akceptujú pohľad na dynamické referencie. Vďaka pravidlám jazyka je do týchto funkcií možné vložiť ľubovoľný argument iného typu, ktorý sa automaticky konvertuje na dynamickú referenciu pomocou tohoto konštruktoru.

Je obmedzený tak, že typ \texttt{r} nemôže byť dynamická referencia, objekt ani premenná.

\subsubsection{\texttt{operator=(const dynamic\_reference\&)}}

Triviálne kopírovacie priradenie. Zmení, na čo referencia ukazuje. Referencie v~\Cpp{} takúto schopnosť nemajú, v~tomto zmysle ide o~hybrid medzi referenciou a~ukazovateľom.

\subsubsection{\texttt{get\_type()}}

Vráti deskriptor implementujúci \texttt{reference\_type}, ktorý označuje typ danej dynamickej referencie.
%$
\begin{code}
int i;
dynamic_reference r = i;
std::cout << r.get_type(); // "int&"
\end{code}

\subsubsection{\texttt{cast\_unsafe(PP::concepts::type auto t)}}

Vráti silno typovanú referenciu na objekt odkazovaný dynamickou referenciou. Ak \texttt{t} reprezentuje referenčný typ, vrátená referencia je toho typu. Ak \texttt{t} reprezentuje hodnotový typ \texttt{T}, vrátená referencia je typu \texttt{T\&\&}.
Interne využíva \texttt{reinterpret\_cast}, takže konverzia je korektná, iba ak vedie na rovnaký typ, akého je interne daná dynamická referencia.
%$
\begin{code}
int i = 4;
dynamic_reference r = i;
std::cout << i.cast_unsafe(PP::type<int&>); // "4"
\end{code}

\subsubsection{\texttt{visit(PP::concepts::type auto t, auto\&\& f)}}

Spustí unárny funktor \texttt{f} s~argumentom typu reprezentovaného \texttt{t} s~pridanými cv a~ref kvalifikátormi na základe dynamickej referencie.
%$
\begin{code}
struct f { void operator()(auto&)  { std::cout << "L"; }
           void operator()(auto&&) { std::cout << "R"; } };

int i;
dynamic_reference r = i;
r.visit(PP::type<int>, f{}); // "L"
r = std::move(i);
r.visit(PP::type<int>, f{}); // "R"
\end{code}

\subsubsection{\texttt{visit\_ptr(PP::concepts::type auto t, auto\&\& f)}}

Podobné ako \texttt{visit}. Predpokladá, že dynamická referencia ukazuje na \texttt{cv T*}, pričom \texttt{t} reprezentuje \texttt{T}. cv sa určí z~dynamickej referencie. Spustí \texttt{f} s~argumentom \texttt{cv T*\&\&}.
%$
\begin{code}
struct f { void operator()(      auto*) { std::cout << "N"; }
           void operator()(const auto*) { std::cout << "C"; } };

int i;
int* p = &i;
dynamic_reference r = p;
r.visit(PP::type<int>, f{}); // "N"
int* p = &std::as_const(i);
r = p;
r.visit(PP::type<int>, f{}); // "C"
\end{code}

\subsection{\texttt{dynamic\_object}}

Dynamický objekt reprezentuje objekt ľubovoľného typu.

Môže sa nachádzať v~troch stavoch: \textit{neplatný}, \texttt{void} a~\textit{platný}. V~stave \textit{neplatný} navyše poskytuje informáciu o~druhu chyby, ktorá nastala. Iba v~stave \textit{platný} obsahuje nejaký objekt.

\subsubsection{Rozhranie}
\begin{code}[fontsize=\footnotesize]
dynamic_object();
dynamic_object(dynamic_object&&);
explicit dynamic_object(PP::concepts::invocable auto&&);
dynamic_object& operator=(dynamic_object&&);
cv_type<complete_object_type> get_cv_type() const;
const complete_object_type& get_type() const;
operator dynamic_reference() const;
explicit operator bool() const;
invalid_code get_error_code() const;
bool is_void() const;

static dynamic_object create_void();
static dynamic_object create_invalid(invalid_code);
static dynamic_object create(PP::concepts::type auto, auto&&...);
\end{code}

\subsubsection{\texttt{dynamic\_object(dynamic\_object\&\& o)}}

Dynamický objekt je výhradným vlastníkom svojej pamäte, takže sa dá iba move konštruovať. Tento konštruktor uvedie objekt \texttt{o} do nedefinovaného stavu, takže sa nedá ďalej použiť.

\subsubsection{\texttt{dynamic\_object(PP::concepts::invocable auto\&\& i)}}

Pre \texttt{X:=PP\_F(i())} tento konštruktor vytvorí dynamický objekt inicializovaný s~\texttt{decltype(X)(X)}.

\subsubsection{\texttt{operator=(dynamic\_object\&\&)}}

Rovnaká sémantika ako má move konštruktor.

\subsubsection{\texttt{operator dynamic\_reference()}}

Vytvorí dynamickú referenciu na dynamický objekt.

\subsubsection{\texttt{operator bool()}}

Vráti \texttt{false} práve vtedy, keď je v~neplatnom stave.

\subsubsection{\texttt{get\_error\_code()}}

Ak je v~neplatnom stave, vráti kód o~stave. Inak vráti \texttt{invalid\_code::none}.

\subsubsection{\texttt{is\_void()}}

Vráti \texttt{true} práve vtedy, keď je v~stave \texttt{void}.

\subsubsection{\texttt{create\_void()}}

Vytvorí dynamický objekt v~stave \texttt{void}.

\subsubsection{\texttt{create\_invalid(invalid\_code code)}}

Vytvorí dynamický objekt v~neplatnom stave s~kódom \texttt{code}.

\subsubsection{\texttt{create(PP::concepts::type auto t, auto\&\&... args)}}

Vytvorí dynamický objekt ako \texttt{PP\_GT(t)(PP\_F(args)...)}.

\subsection{\texttt{dynamic\_variable}}

Reprezentuje dynamickú premennú, sum typ dynamickej referencie a~dynamického objektu.

\subsubsection{Rozhranie}

\begin{code}[fontsize=\footnotesize]
explicit dynamic_variable(dynamic_reference);
explicit dynamic_variable(dynamic_object&&);
dynamic_variable(dynamic_variable&&);
dynamic_variable& operator=(dynamic_variable&&);
explicit operator bool() const;
dynamic_object::invalid_code get_error_code() const;
cv_type<type> get_type() const;
operator dynamic_reference() const;

static dynamic_variable create_void();
static dynamic_variable create_invalid(dynamic_object::invalid_code);
static dynamic_variable create(auto&&);
\end{code}

\subsubsection{\texttt{dynamic\_variable(dynamic\_reference)}}

Vytvorí dynamickú premennú reprezentujúcu referenciu.

\subsubsection{\texttt{dynamic\_variable(dynamic\_object\&\&)}}

Vytvorí dynamickú premennú reprezentujúcu objekt.

\subsubsection{\texttt{dynamic\_variable(dynamic\_variable\&\& v)}}

Vykradne \texttt{v}. Ak \texttt{v} obsahovalo dynamickú referenciu, ostane nezmenené. Ak obsahovalo dynamický objekt, daný objekt sa presunie.

\subsubsection{\texttt{operator=(dynamic\_variable\&\&)}}

Rovnaká sémantika ako pri move konštruktore.

\subsubsection{\texttt{operator bool()}}

Ak obsahuje dynamickú referenciu, vráti \texttt{true}. Ak obsahuje dynamický objekt, vráti \texttt{operator bool} dynamického objektu.

\subsubsection{\texttt{get\_error\_code()}}

Ak obsahuje dynamickú referenciu, vráti \texttt{invalid\_code::none}. Ak obsahuje dynamický objekt, vráti chybový kód dynamického objektu.

\subsubsection{\texttt{get\_type()}}

Vráti dvojicu cv a~typ. Ak obsahuje dynamickú referenciu na \texttt{T\&\&}, vráti \texttt{(cv::none, T\&\&)}. Ak obsahuje dynamický objekt typu \texttt{cv T}, vráti \texttt{(cv, T)}.

\subsubsection{\texttt{operator dynamic\_reference()}}

Vráti kópiu dynamickej referencie, ak obsahuje referenciu. Ak obsahuje objekt, vráti dynamickú referenciu naň. Táto funkcia poskytuje prístup ku objektu, ktorý reprezentuje dynamická premenná.

\subsubsection{\texttt{create\_void()}}

Vytvorí dynamickú premennú obsahujúcu dynamický objekt vo \texttt{void} stave.

\subsubsection{\texttt{create\_invalid(dynamic\_object::invalid\_code code)}}

Vytvorí dynamickú premennú obsahujúcu dynamický objekt v~neplatnom stave s~kódom \texttt{code}.

\subsubsection{\texttt{create(auto\&\& f)}}

\texttt{f} musí byť funktor zavolateľný bez argumentov.

Ak \texttt{f} vracia referenčný typ, vytvorí sa dynamická premenná obsahujúca dynamickú referenciu na návratovú hodnotu volania funktoru.

Ak vracia hodnotový typ, vytvorí sa dynamická premenná obsahujúca dynamický objekt s~hodnotou volania funktoru.

Ak vracia \texttt{void}, vytvorí sa dynamická premenná obsahujúca dynamický objekt v~stave \texttt{void}.

\subsection{\texttt{candidate\_functions}}

Kandidáti reprezentujú množinu funkcií, ktorá sa dá zavolať. Volanie funkcie je v~\Cpp{} definované v~skutočnosti na množine, takže táto trieda dáva to správne rozhranie pre implementáciu korektného chovania podľa štandardu.

Trieda navyše spĺňa koncept pohľadu (na \texttt{const function\&}).

\subsubsection{Rozhranie}

\begin{code}[fontsize=\footnotesize]
using IL = std::initializer_list<dynamic_reference>;

explicit candidate_functions(PP::concepts::view auto&&);
candidate_functions& trim_by_name(PP::string_view);
candidate_functions& trim_by_exact_argument_count(PP::size_t);
dynamic_variable invoke(const IL&) const;
dynamic_variable invoke(PP::concepts::view auto&&) const;
\end{code}

\subsubsection{\texttt{candidate\_functions(PP::concepts::view auto\&\&)}}

Inicializuje kandidátov z~pohľadu na deskriptory funkcií.

\subsubsection{\texttt{trim\_by\_name(PP::string\_view name)}}

Oreže z~množiny všetky funkcie, ktoré sa nevolajú \texttt{name}.

\subsubsection{\texttt{trim\_by\_exact\_argument\_count(PP::size\_t count)}}

Oreže z~množiny všetky funkcie, ktoré majú iný počet parametrov ako \texttt{count}.

\subsubsection{\texttt{invoke(PP::concepts::view auto\&\&)}}

Ako argument akceptuje pohľad na dynamické referencie na argumenty volania. Interne spustí overload resolution. Ak OR zlyhá, vráti dynamickú premennú obsahujúcu dynamický objekt v~neplatnom stave s~príslušným kódom o~chybe. Ak nezlyhá, vráti výsledok volania vybranej funkcie.

\subsubsection{\texttt{invoke(const IL\&)}}

Zavolá predošlý \texttt{invoke}. Poskytuje možnosť použiť inicializáciu so zloženými zátvorkami.

\subsection{\texttt{viable\_functions}}

Realizovateľné funkcie. Reprezentujú množinu funkcií s~už spustením overload resolution a~pripravenými konverznými sekvenciami. Rozhraním sú podobné ako kandidáti, ide predovšetkým o~optimalizáciu.

\subsubsection{Rozhranie}

\begin{code}[fontsize=\footnotesize]
using IL = std::initializer_list<dynamic_reference>;
template <typename T> concept v~= PP::concepts::view<T>;
template <typename T> concept T = PP::concepts::tuple<T>;

viable_functions(V auto&&, v~auto&&);
viable_functions(V auto&&, T auto&&...);
dynamic_variable invoke(V auto&& arguments) const;
dynamic_variable invoke(const IL& arguments) const;
\end{code}

\subsubsection{\texttt{viable\_functions(V auto\&\& functions, v~auto\&\& argument\_lists)}}

Akceptuje pohľad na deskriptory funkcií a~pohľad na pohľady na deskriptory referenčných typov. Druhý parameter reprezentuje množinu všetkých možných postupností typov argumentov, ktoré chce užívateľ pri volaní používať.

Inicializuje množinu s~\texttt{functions} a~spustí overload resolution pre každý z~pohľadov na referenčné typy.

\subsubsection{\texttt{viable\_functions(V auto\&\& functions, T auto\&\&... argument\_tuples)}}

Poskytuje príjemnejšie rozhranie ako predošlý konštruktor. Ako druhý argument akceptuje n-ticu n-tíc referenčných typov.

\subsubsection{\texttt{dynamic\_variable invoke(V auto\&\& arguments) const}}

Zavolá funkciu s~\texttt{arguments}. Funkcia sa nevyberie pomocou OR, ale iba pomocou porovnania s~postupnosťami typov argumentov, ktoré boli pri konštrukcii poskytnuté.

\subsubsection{\texttt{dynamic\_variable invoke(const IL\& arguments) const}}

Zavolá predošlý \texttt{invoke}. Poskytuje možnosť použiť inicializáciu so zloženými zátvorkami.

\section{Demonštrácia}

V tejto časti vytvoríme program demonštrujúci schopnosti knižnice. V~prílohe sa tento projekt nachádza pod názvom \texttt{Demo2}.

Vytvoríme \uv{polovičný} visitor návrhový vzor. Predpokladajme nasledujúce triedy:
%$
\begin{code}
struct B { virtual ~B() = default; };
struct D1 : B {};
struct D2 : B {};
\end{code}

K nim vytvoríme visitor:
%$
\begin{code}
struct visitor
{   void visit(B&);
    virtual void visit(D1&) = 0;
    virtual void visit(D2&) = 0; };
\end{code}

V inom súbore vytvoríme implementáciu visitoru. Pre túto ukážku napíšeme do definície triedy aj definície funkcií.
%$
\begin{code}
struct printing_visitor : visitor
{   std::ostream& out;
public:
    printing_visitor(std::ostream& out) : out(out) {}
    using visitor::visit;
    void visit(D1&) override { out << "D1"; }
    void visit(D2&) override { out << "D2"; } };
\end{code}

Potom môžeme vytvoriť \texttt{main}, v~ktorom použijeme konkrétnu implementáciu \texttt{visitor}, ale referencie, ktoré máme k~dispozícii sú na typ \texttt{B}.
%$
\begin{code}
std::unique_ptr<B> f(bool x)
{   if (x) return std::make_unique<D1>();
    else   return std::make_unique<D2>();
}
int main()
{   printing_visitor v(std::cout);
    v.visit(*f(true));  // "D1"
    v.visit(*f(false)); // "D2"
    return 0;
}
\end{code}

Úlohou je už len implementovať \texttt{visitor::visit(B\&)}. Použijeme upcast na dynamický typ. Na volanie použijeme \texttt{candidate\_functions}, navyše v~kombinácii s~\texttt{viable\_functions} pre väčšiu optimalizáciu.
%$
\begin{code}[fontsize=\footnotesize]
void visitor::visit(B& b)
{
    static const auto viables = PPreflection::viable_functions(
        PPreflection::candidate_functions(
            Preflection::type::reflect(PP::type<visitor>)
                .get_member_functions())
            .trim_by_name("visit"),
        PP::type_tuple<visitor&, D1&>,
        PP::type_tuple<visitor&, D2&>);

    viables.invoke({*this, PPreflection::dynamic_polymorphic_reference(b)});
}
\end{code}
