\chapter{Súvisiace práce} \label{suvisiace}

V tejto kapitole preskúmame existujúce implementácie introšpekcie v~\Cpp{}. Cieľ je tiež ukázať \uv{medzeru} v~ich schopnostiach, ktorú \PPreflection{} zapĺňa. Časťou analýzy každej knižnice bude príklad jej použitia. Pri každom príklade predpokladáme existenciu nasledovnej triedy:

\begin{code}
struct Struct {
    int field;
    S() : field(7) {};
    void method(double) {};
};
\end{code}

\section{\textsf{RTTR}}

Knižnica \textsf{RTTR} ponúka introšpekciu s~manuálnou registráciou. Pri tejto manuálnej registrácii musí užívateľ explicitne vymenovať všetky entity, ktoré chce do introšpekcie registrovať. Dokáže napríklad enumerovať členov, vytvárať dynamické objekty, dynamicky volať funkcie a~pristupovať ku dátovým členom skrz reflektované entity.

Dynamické volania vracajú typ \texttt{variant}, ktorý môže obsahovať objekt ľubovoľného typu. Nie sú pri nich povolené žiadne konverzie --- typy parametrov a~argumentov sa musia zhodovať. Tiež pri nich nie je možné použiť \texttt{variant} ako argument.

Autor: Axel Menzel, \url{https://github.com/rttrorg/rttr}.

\subsection*{Príklad}

\begin{code}
#include <rttr/registration>

RTTR_REGISTRATION
{   rttr::registration::class_<Struct>("Struct")
         .property("field", &Struct::field)
         .constructor<>()
         .method("f", &Struct::method);
}
// ...
Struct s;
auto t = type::get<Struct>();
auto field = t.get_property("field");
std::cout << field.get_value(s).get_value<int>(); // vypíše 7
\end{code}

\section{\textsf{meta}}

Táto knižnica nepoužíva makrá. Namiesto toho vytvára objekty \texttt{factory}, do ktorých sa manuálne registrujú informácie o~entite. Má podobné schopnosti ako \textsf{RTTR}. Dynamické volania vracajú typ \texttt{any}, ktorý --- na rozdiel od \textsf{RTTR} --- je možné použiť ako argument. Takisto ako predošlá knižnica nepodporuje konverzie, typy sa pri volaní musia zhodovať.

Autor: Michele Caini, \url{https://github.com/skypjack/meta}.

\subsection*{Príklad}

\begin{code}
#include <meta/factory.hpp>
#include <meta/meta.hpp>

std::hash<std::string_view> hash{};

meta::reflect<Struct>(hash("Struct"))
    .data<&Struct::field>(hash("field"));
    .ctor<>()
    .func<&Struct::method>(hash("method"));
    
Struct s;
auto t = meta::resolve<Struct>();
auto field = t.data(hash("field"));
std::cout << field.get(s).cast<int>(); // vypíše 7
\end{code}

\section{\textsf{SelfPortrait}}

Táto knižnica ponúka generátor metadát, takže nie je potrebná explicitná registrácia entít. Inak je schopnosťami porovnateľná s~\textsf{meta}. Tiež neimplementuje konverzie pri dynamickom volaní.

Autori: \citet{bayser2012system}, \url{https://github.com/maxdebayser/SelfPortrait}.

\subsection*{Príklad}

\begin{code}
VariantValue s = Struct();
auto t = Class::lookup("Struct");
auto field = t.getAttribute("field");
std::cout << field.get(s).value<int&>(); // vypíše 7
\end{code}

\section{Ostatné}

\begin{itemize}
    \item \textsf{Ponder} - používa makrá, nutná explicitná registrácia
    
    \url{https://github.com/billyquith/ponder}
    
    \item \textsf{Refureku} - nutné použitie makier na registráciu na mieste deklarácie entít
    
    \url{https://github.com/jsoysouvanh/Refureku}
    
    \item \textsf{CPP-Reflection} - zdanlivo populárna knižnica, no zdá sa byť nedokončená
    
    \url{https://github.com/AustinBrunkhorst/CPP-Reflection}
    
\end{itemize}

\section{Zhrnutie}

Použitie makier je medzi knižnicami pre introšpekciu populárny spôsob ako generovať metadáta. Žiaľ, v~istom zmysle popiera princíp introšpekcie ako schopnosti jazyka. Ak musí užívateľ explicitne registrovať entity jazyka do knižnice, nejde potom o~schopnosť \emph{jazyka} reflektovať na svoju štruktúru. Užitočnosť týchto knižníc to samozrejme nepopiera --- nutnosť registrovať entity sa môže aj tak vyplatiť --- no poukazuje to na silný nedostatok. Tiež sme neboli schopní nájsť ani jednu knižnicu, ktorá by sa aspoň snažila mechanizmus volania reflektovaných funkcií priblížiť tomu zo štandardného \Cpp{}, teda implementovala by implicitné konverzie.

\begin{table}[htbp]
\centering
\begin{tabular}{l@{\hspace{0.4cm}}ccc}
\toprule
\textbf{Názov projektu} & \mc{\textbf{Bez makier}} & \mc{\textbf{Automatická registrácia}} & \mc{\textbf{Konverzie}} \\
\midrule
Ponder          & \textbf{-} & \textbf{-} & \textbf{-} \\
Refureku        & \textbf{-} & \textbf{-} & \textbf{-} \\
RTTR            & \textbf{-} & \textbf{-} & \textbf{-} \\
meta            & \checkmark & \textbf{-} & \textbf{-} \\
SelfPortrait    & \checkmark & \checkmark & \textbf{-} \\
\PPreflection{} & \checkmark & \checkmark & \checkmark \\
\bottomrule
\end{tabular}
\caption{Porovnanie knižníc na introšpekciu v~\Cpp{}}
\label{tab1}
\end{table}
