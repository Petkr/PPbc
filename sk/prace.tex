%%% Hlavní soubor. Zde se definují základní parametry a odkazuje se na ostatní části. %%%

%% Verze pro jednostranný tisk:
% Okraje: levý 40mm, pravý 25mm, horní a dolní 25mm
% (ale pozor, LaTeX si sám přidává 1in)
\documentclass[12pt,a4paper]{report}
\setlength\textwidth{145mm}
\setlength\textheight{247mm}
\setlength\oddsidemargin{15mm}
\setlength\evensidemargin{15mm}
\setlength\topmargin{0mm}
\setlength\headsep{0mm}
\setlength\headheight{0mm}
% \openright zařídí, aby následující text začínal na pravé straně knihy
\let\openright=\clearpage

%% Pokud tiskneme oboustranně:
% \documentclass[12pt,a4paper,twoside,openright]{report}
% \setlength\textwidth{145mm}
% \setlength\textheight{247mm}
% \setlength\oddsidemargin{14.2mm}
% \setlength\evensidemargin{0mm}
% \setlength\topmargin{0mm}
% \setlength\headsep{0mm}
% \setlength\headheight{0mm}
% \let\openright=\cleardoublepage

%% Vytváříme PDF/A-2u
\usepackage[a-2u]{pdfx}

%% Přepneme na českou sazbu a fonty Latin Modern
\usepackage[slovak]{babel}
\usepackage{lmodern}
\usepackage[T1]{fontenc}
\usepackage{textcomp}

%% Použité kódování znaků: obvykle latin2, cp1250 nebo utf8:
\usepackage[utf8]{inputenc}

%%% Další užitečné balíčky (jsou součástí běžných distribucí LaTeXu)
\usepackage{amsmath}        % rozšíření pro sazbu matematiky
\usepackage{amsfonts}       % matematické fonty
\usepackage{amsthm}         % sazba vět, definic apod.
\usepackage{bbding}         % balíček s nejrůznějšími symboly
			    % (čtverečky, hvězdičky, tužtičky, nůžtičky, ...)
\usepackage{bm}             % tučné symboly (příkaz \bm)
\usepackage{graphicx}       % vkládání obrázků
\usepackage{fancyvrb}       % vylepšené prostředí pro strojové písmo
\usepackage{indentfirst}    % zavede odsazení 1. odstavce kapitoly
\usepackage{natbib}         % zajištuje možnost odkazovat na literaturu
			    % stylem AUTOR (ROK), resp. AUTOR [ČÍSLO]
\usepackage[nottoc]{tocbibind} % zajistí přidání seznamu literatury,
                            % obrázků a tabulek do obsahu
\usepackage{icomma}         % inteligetní čárka v matematickém módu
\usepackage{dcolumn}        % lepší zarovnání sloupců v tabulkách
\usepackage{booktabs}       % lepší vodorovné linky v tabulkách
\usepackage{paralist}       % lepší enumerate a itemize
\usepackage{xcolor}         % barevná sazba

%%% Údaje o práci

% Název práce v jazyce práce (přesně podle zadání)
\def\NazevPrace{Systém pro introspekci v C++}

% Název práce v angličtině
\def\NazevPraceEN{A system for introspection in C++}

% Jméno autora
\def\AutorPrace{Peter Fačko}

% Rok odevzdání
\def\RokOdevzdani{2021}

% Název katedry nebo ústavu, kde byla práce oficiálně zadána
% (dle Organizační struktury MFF UK, případně plný název pracoviště mimo MFF)
\def\Katedra{Katedra softwarového inženýrství}
\def\KatedraEN{Department of Software Engineering}

% Jedná se o katedru (department) nebo o ústav (institute)?
\def\TypPracoviste{Katedra}
\def\TypPracovisteEN{Department}

% Vedoucí práce: Jméno a příjmení s~tituly
\def\Vedouci{RNDr. David Bednárek, Ph.D.}

% Pracoviště vedoucího (opět dle Organizační struktury MFF)
\def\KatedraVedouciho{Katedra softwarového inženýrství}
\def\KatedraVedoucihoEN{Department of Software Engineering}

% Studijní program a obor
\def\StudijniProgram{Informatika}
\def\StudijniObor{Programování a softwarové systémy}

% Nepovinné poděkování (vedoucímu práce, konzultantovi, tomu, kdo
% zapůjčil software, literaturu apod.)
\def\Podekovani{%
Ďakujem vedúcemu tejto práce RNDr. Davidovi Bednárkovi, Ph.D. za ochotu a čas, ktorý mi pri písaní tejto práce venoval.

Ďalej ďakujem svojej rodine a priateľom za podporu počas celého štúdia.
}

% Abstrakt (doporučený rozsah cca 80-200 slov; nejedná se o zadání práce)
\def\Abstrakt{%
Introspekcia a reflexia sú silné programovacie nástroje umožňujúce všeobecné riešenie problematických úloh, ako napríklad multiple dispatch alebo serializácia. V C++ je podpora pre introspekciu veľmi slabá a~reflexia nie je možná vôbec. Cieľom tejto práce je vytvoriť knižnicu pre introspekciu, pracujúcu s~metadátami vygenerovanými zo zdrojového textu, a~vytvoriť generátor týchto metadát ako nástroj nad prekladačom. Funkčnosť celého systému demonštrujeme implementáciou multiple dispatch, ktorý nie je implementovateľný v~štandardnom C++.
}
\def\AbstraktEN{%
Introspection and reflection are powerful programming tools that allow generic solutions for difficult problems such as multiple dispatch or serialization. In C++, the support for introspection is seriously limited, while reflection is virtually non-existent. The goal of this thesis is the creation of an introspection library working with metadata generated from source code and the creation of a compiler-based metadata generator. The functionality of the whole system shall be demonstrated by implementing multiple dispatch which is not implementable in standard C++.
}

% 3 až 5 klíčových slov (doporučeno), každé uzavřeno ve složených závorkách
\def\KlicovaSlova{%
{introspekcia} {C++} {analýza zdrojových textov}
}
\def\KlicovaSlovaEN{%
{introspection} {C++} {source-code analysis}
}

%% Balíček hyperref, kterým jdou vyrábět klikací odkazy v PDF,
%% ale hlavně ho používáme k uložení metadat do PDF (včetně obsahu).
%% Většinu nastavítek přednastaví balíček pdfx.
\hypersetup{unicode}
\hypersetup{breaklinks=true}

%% Definice různých užitečných maker (viz popis uvnitř souboru)
\include{makra}

%% Titulní strana a různé povinné informační strany
\begin{document}
\include{titulka}

%%% Strana s automaticky generovaným obsahem bakalářské práce

\tableofcontents

%%% Jednotlivé kapitoly práce jsou pro přehlednost uloženy v samostatných souborech
\include{uvod}
\chapter{Definície}

Na začiatok uvedieme definície pojmov, ktoré sú pre túto prácu bazálne.
\emph{Reflexia} je schopnosť programu manipulovať s reprezentáciami stavu daného programu ako s dátami.
Jeden z aspektov reflexie je \emph{introspekcia}, čo je schopnosť programu pozorovať svoj vlastný stav.

\chapter{Význam reflexie a introspekcie}



\chapter{Introspekcia v C++}

C++ ponúka introspekciu v dvoch podobách.

Do prvej spadajú operátory \texttt{sizeof}, \texttt{sizeof...} a \texttt{alignof}. Napríklad, \texttt{sizeof} akceptuje ako argument typ alebo výraz a vracia veľkosť daného typu, respektíve veľkosť typu daného výrazu. Ak je argument výraz, zvažuje sa jeho \emph{statický typ}, nie \emph{dynamický typ}. Statický typ je typ plynúci z deklarácie v zdrojovom texte, napríklad pri výraze volania funkcie je to návratový typ funkcie. Dynamický typ výrazu je typ najviac odvodeného objektu, na ktorý výraz odkazuje.
%$
\begin{code}
struct Base {};
struct Derived : Base {};

Base b;
Derived d;

Base& x = b;
Base& y = d;

x; // static type: Base, dynamic type: Base
y; // static type: Base, dynamic type: Derived
\end{code}
%$
\texttt{sizeof} svoj argument nevyhodnocuje - entita, ktorú skúma, je entita zdrojového textu, nie hodnota v pamäti. Ostatné dva operátory fungujú obdobne: nezvažujú stav programu počas behu, ale informácia, ktorú ponúkajú, plynie iba z formy zdrojového textu. Táto podoba introspekcie pozoruje vlastnosti programu v kompilačnom čase, nazveme ju teda \emph{kompilačná introspekcia}.

Druhá podoba introspekcie v C++ je založená na operátore \texttt{typeid}. Ten akceptuje typ alebo výraz a vracia odkaz na \texttt{std::type\_info}, ktorý reprezentuje typ, respektíve typ výrazu. \texttt{std::type\_info} má definované úplné usporiadanie a dokáže získať meno reprezentovaného typu. Rozdiel oproti kompilačnej introspekcii je tu v tom, že \texttt{typeid} svoj argument vyhodnocuje, pretože ako typ výrazu rozumie jeho dynamický typ. Ten je známy až za behu z hodnoty výrazu. Táto introspekcia pozoruje program za behu, nazveme ju \emph{behová introspekcia}.

Tento pohľad na druhy introspekcie v C++ ukazuje istú medzeru v ponúkaných nástrojoch štandardným C++. Ak máme spôsob, ako z typu určiť jeho veľkosť, a tiež máme nástroj, ako z hodnoty určiť jej \uv{behový} typ, prečo nemožno jednoducho určiť z hodnoty veľkosť jej behového typu? To by vyžadovalo mapovanie medzi reprezentantmi typov a informáciami o typoch. K tomuto problému sa vrátime pri implementácii.

V tejto kapitole sú zámerne vynechané detaily, ktoré nie sú podstatné pre účel tejto časti. Špeciálne: dynamický typ má zmysel iba pre výrazy kategórie glvalue, prvalue sa v \texttt{typeid} nevyhodnocuje a  \texttt{typeid} rozlišuje iba dynamický typ výrazov polymorfných typov. Všetky tieto nepresnosti budú v práci vyjasnené, keď budú mať nejaký dopad.


\include{zaver}

%%% Seznam použité literatury
\include{literatura}

%%% Obrázky v bakalářské práci
%%% (pokud jich je malé množství, obvykle není třeba seznam uvádět)
\listoffigures

%%% Tabulky v bakalářské práci (opět nemusí být nutné uvádět)
%%% U matematických prací může být lepší přemístit seznam tabulek na začátek práce.
\listoftables

%%% Použité zkratky v bakalářské práci (opět nemusí být nutné uvádět)
%%% U matematických prací může být lepší přemístit seznam zkratek na začátek práce.
\chapwithtoc{Zoznam použitých skratiek}

%%% Přílohy k bakalářské práci, existují-li. Každá příloha musí být alespoň jednou
%%% odkazována z vlastního textu práce. Přílohy se číslují.
%%%
%%% Do tištěné verze se spíše hodí přílohy, které lze číst a prohlížet (dodatečné
%%% tabulky a grafy, různé textové doplňky, ukázky výstupů z počítačových programů,
%%% apod.). Do elektronické verze se hodí přílohy, které budou spíše používány
%%% v elektronické podobě než čteny (zdrojové kódy programů, datové soubory,
%%% interaktivní grafy apod.). Elektronické přílohy se nahrávají do SISu a lze
%%% je také do práce vložit na CD/DVD. Povolené formáty souborů specifikuje
%%% opatření rektora č. 72/2017.
\appendix
\chapter{Prílohy}

\section{Prvá príloha}

\openright
\end{document}
